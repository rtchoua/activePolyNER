\documentclass[conference]{IEEEtran}

% my packages
\usepackage{url}
\usepackage{textcomp}
\usepackage{listings}
\usepackage[T1]{fontenc} % To use quotes in listing
% \usepackage{fixltx2e}
\usepackage{subfigure,epsfig,amsfonts}
\usepackage{caption}
\usepackage{xspace}
\usepackage[usenames,dvipsnames,svgnames]{xcolor}
\usepackage{multicol}
\usepackage{graphicx}
\definecolor{codegreen}{rgb}{0,0.6,0}
\definecolor{codegray}{rgb}{0.5,0.5,0.5}
\definecolor{codepurple}{rgb}{0.58,0,0.82}
\definecolor{backcolour}{rgb}{0.95,0.95,0.92}
\usepackage{siunitx}
\usepackage{array}
\newcolumntype{P}[1]{>{\centering\arraybackslash}p{#1}}

\usepackage[font={small,bf}]{caption}
\usepackage{balance} 
\usepackage{flexisym}
\usepackage{varwidth}
\usepackage{makecell}
%\usepackage{lipsum}
\lstdefinestyle{mystyle}{
    backgroundcolor=\color{backcolour}, 
    commentstyle=\color{codegreen},
    keywordstyle=\color{magenta},
    numberstyle=\tiny\color{codegray},
    stringstyle=\color{codepurple},
    basicstyle=\footnotesize\ttfamily,
    breakatwhitespace=false,         
    breaklines=true,                 
    captionpos=b,                    
    keepspaces=true,                 
    numbers=left,                    
    numbersep=5pt,                  
    showspaces=false,                
    showstringspaces=false,
    showtabs=false,                  
    tabsize=2
}
\lstset{style=mystyle}



\newcommand{\easychair}{\textsf{easychair}}
\newcommand{\miktex}{MiK{\TeX}}
\newcommand{\texniccenter}{{\TeX}nicCenter}
\newcommand{\makefile}{\texttt{Makefile}}
\newcommand{\latexeditor}{LEd}

\newcommand{\CHIDB}{$\chi$DB\xspace}
\newcommand{\CHI}{$\chi$\xspace}
\newcommand{\TG}{\emph{T}$\!_\emph{g}$}

\usepackage[noadjust]{cite}
\renewcommand{\citepunct}{,\penalty\citepunctpenalty\,}
\renewcommand{\citedash}{--}% optionally

\newcommand{\veryshortarrow}[1][3pt]{$\mathrel{%
   \hbox{\rule[\dimexpr\fontdimen22\textfont2-.2pt\relax]{#1}{.4pt}}%
   \mkern-4mu\hbox{\usefont{U}{lasy}{m}{n}\symbol{41}}}$}

\newif\ifdraft
%\draftfalse
\drafttrue

\ifdraft
  \newcommand{\ian}[1]{\textcolor{Red}{[Ian: #1]}\xspace}
  \newcommand{\roselyne}[1]{\textcolor{Blue}{[Roselyne: #1]}\xspace}
  \newcommand{\kyle}[1]{\textcolor{Purple}{[Kyle: #1]}\xspace}
  \newcommand{\logan}[1]{\textcolor{Green}{[Logan: #1]}\xspace}
  \newcommand{\debbie}[1]{\textcolor{Brown}{[Debbie: #1]}\xspace}
  \newcommand{\shrayesh}[1]{\textcolor{BurntOrange}{[Shrayesh: #1]}\xspace}
  \newcommand{\aswathy}[1]{\textcolor{WildStrawberry}{[Aswathy: #1]}\xspace}
  \newcommand{\hong}[1]{\textcolor{RoyalPurple}{[Hong: #1]}\xspace}
\else
  \newcommand{\ian}[1]{}
  \newcommand{\roselyne}[1]{}
  \newcommand{\kyle}[1]{}
  \newcommand{\logan}[1]{}
  \newcommand{\debbie}[1]{}
  \newcommand{\shrayesh}[1]{}
  \newcommand{\debbie}[1]{}
  \newcommand{\shrayesh}[1]{}
  \newcommand{\aswathy}[1]{}
  \newcommand{\hong}[1]{}
\fi

\newif\ifnist
\nistfalse
%\nisttrue

\ifnist
  \newcommand\nistnum[1]{\num[group-separator={},mode=text]{#1}}
\else
  \newcommand\nistnum[1]{\num[group-minimum-digits=4,group-separator={,},mode=text]{#1}} 
\fi

% correct bad hyphenation here
\hyphenation{op-tical net-works semi-conduc-tor}

\begin{document}
% This line plus the intro lines in the Bib file concatenates author lists.
\bstctlcite{IEEEexample:BSTcontrol}

%\title{PolyNER: A \textit{word2vec} Approach to Scientific NER}
\title{Generating Training Data for Scientific Named Entity with Minimal Human Effort}


\author{\IEEEauthorblockN{Roselyne B. Tchoua\IEEEauthorrefmark{1},
Zhi Hong\IEEEauthorrefmark{1}, 
Logan T. Ward\IEEEauthorrefmark{2}\IEEEauthorrefmark{3}, \\
Kyle Chard\IEEEauthorrefmark{2}\IEEEauthorrefmark{3},
Debra J. Audus\IEEEauthorrefmark{4}, 
Shrayesh N. Patel\IEEEauthorrefmark{5}, 
Juan J. de Pablo\IEEEauthorrefmark{5} and
Ian T. Foster\IEEEauthorrefmark{1}\IEEEauthorrefmark{2}\IEEEauthorrefmark{3}}
\IEEEauthorblockA{\IEEEauthorrefmark{1}Department of Computer Science, University of Chicago, Chicago, IL, USA\\
Email: roselyne@uchicago.edu}
\IEEEauthorblockA{\IEEEauthorrefmark{2}Globus, University of Chicago, Chicago, IL, USA\\}
\IEEEauthorblockA{\IEEEauthorrefmark{3}Data Science and Learning Division, Argonne National Laboratory, Argonne, IL, USA\\}
\IEEEauthorblockA{\IEEEauthorrefmark{4}Materials Science and Engineering Division, National Institute of Standards and Technology, Gaithersburg, MD, USA\\}
\IEEEauthorblockA{\IEEEauthorrefmark{5}Institute for Molecular Engineering, University of Chicago, Chicago, IL, USA}
}

% make the title area
\maketitle

\begin{abstract}

Despite significant progress in natural language processing, 
machine learning models require substantial expert-annotated data for training to perform well in tasks such as named entity recognition and entity relations extraction.
Scientific Named Entity Referent Extraction is often more complicated than traditional Named Entity Recognition (NER). 
For example, in polymer science, 
chemical structure may be encoded in a variety of nonstandard
naming conventions,
the same concept can be expressed using
any number of different terms (synonymy),
and authors may refer to polymers with ad-hoc labels (in lieu of longer names).
This adds a layer of difficulty when generating training data for new domains as specialized skills are sometimes needed to correctly label text.
Here we propose polyNER: a semi-automated system for efficient identification of scientific entities in text.
PolyNER applies word embedding models
to generate entity-rich corpora for productive expert labeling,
and then uses the resulting labeled data to 
bootstrap a context-based word vector classifier.
We use approximately labeled candidates and active learning to efficiently obtain annotations from experts, a process that is otherwise tedious and expensive.
With under ten hours of expert time, polyNER demonstrates discrimination capacity comparable to state-of-the-art chemical NLP toolkit and
highlights the potential for human-computer partnership for 
constructing domain-specific scientific NER systems.

\end{abstract}

\begin{IEEEkeywords}
Named Entity Recognition, Machine Learning, Word Embedding, Active Learning, Polymers
\end{IEEEkeywords}

\IEEEpeerreviewmaketitle

%------------------------------------------------------------------------------
\section{Introduction}
\label{sect:introduction}
Despite the large number of publications in this domain, the process of designing new materials is still one of trial and
error.
The field of materials informatics aims to address this problem by combining large datasets and computational models to achieve targeted materials design and reduce time-to-market and development costs of new materials.
A large amount of such data already exists in unstructured text in the scientific literature, hence the need for materials information extraction.

There is a considerable amount of prior work in scientific facts extraction, notably in biomedicine. 
State-of-the-art methods often use hybrid rule-based, machine learning and statistical techniques to extract entity names and relations from the literature~\cite{leaman2008banner,zeng2015survey}. 
While similar efforts have begun in materials science~\cite{hawizy2011chemicaltagger,rocktaschel2012chemspot,leaman2015tmchem,swain2016chemdataextractor}, the lack of available training data for new materials impedes rapid progress. 
Instead, attempts to extract a new type of materials rely  on large, carefully annotated training
data tailored for this new target, often requiring some amount of in-depth domain knowledge.

To address these challenges, we propose polyNER, a system for generating training data for NER using semi-supervised learning and focused expert input. 
PolyNER uses natural language processing to produce enrich sets of candidate entities;
experts then iteratively approve or reject candidates in small batches generated via random sampling or active learning;
the labels are used to train an ensemble of word vector classifiers after each iteration.
The goal is
to substitute the labor-intensive processes of assembling a large
manually annotated corpus (and save cost) by using small numbers of carefully selected candidates to be labeled. 
We use active learning with maximum entropy uncertainty sampling and two different pools of unlabeled data to train classifiers and compare their learning rate after 5 iterations. 
Using all generated labels we train a final word vector classifier and achieve performance comparable to  a state-of-the art rule-based chemical entity extraction
system, ChemDataExtractor (CDE)~\cite{swain2016chemdataextractor}, which we have previously enhanced
with dictionary- and rule-based methods for identifying polymers~\cite{tchoua2017towards}.
Our system however, took less than 5 hours of expert time to achieve this result.

The rest of this paper is as follows. 
In Section~\ref{sect:background}, we motivate the need for identifying polymer names in
text. 
We review semi-supervised methods for NLP systems in
Section~\ref{sect:relatedwork}. 
We describe the design and implementation in Section~\ref{sect:architecture} and evaluate polyNER
in Section~\ref{sect:results}. We summarize and discuss future work in Section~\ref{sect:conclusion}.
%------------------------------------------------------------------------------

%------------------------------------------------------------------------------
\section{Motivation}
\label{sect:background}
\roselyne{Adjust/shorten since I added text.}
\kyle{Sounds similar to previous paper. First paragraph should be re-written.}

The complexity of scientific NER is primarily due to the fact that entities,
both biological~\cite{kim2004introduction} and chemical~\cite{krallinger2015chemdner}, can be described in different ways. 
Subfields of materials science, including polymer science that we focus on here, involve the same variability in naming conventions.
In principle, the International Union of Pure and Applied Chemistry (IUPAC) guidelines define two 
methods for how a polymer should be named~\cite{hiorns2013brief}.
However, such standards are not always followed in practice~\cite{tamames2006success}.
Polymer names may be reported as source-based names (based on the monomer name), structure-based names (based on the repeat unit), common names (requiring domain specific knowledge), trade names (based on the manufacturer), and names based on some of chemical groups within the polymer (requiring context to fully specify the chemistry).
Other naming variations result from typographical variants (e.g., alternative uses of hyphens, brackets, spacing) and alternative component orders.
The origin of these different naming conventions is closely linked to the desire for clarity within a journal article coupled with the often complicated monomeric structures~\cite{audus2017polymer}.
For example, sequence-defined polymers, where multiple monomers are chemically bound in a well-defined sequence as in proteins, often defy normal naming practices as it is not possible to concisely list every monomer and the respective position of the monomer~\cite{lutz2014polymer}. 
%Another class of polymers that often suffer from complicated names are conjugated polymers, which exhibit useful optical and electrical properties. 
%Conjugated polymers are complex due to the co-polymerization of multiple monomers (donor/acceptor units), the type and position of side chains along the polymer backbone and the coupling between monomer units to control regioregularity~\cite{himmelberger2015engineering}.

In addition to challenges related to the makeup of the scientific entities, the paucity of entities in scientific literature and the lack of training data also impedes the use of recent machine learning based NER techniques.
Considerable time and manual effort are involved in creating and maintaining the balanced
Conference on Computational Natural Language Learning (CoNLL) dataset for standard NER~\cite{tjong2003introduction}.
Example sentences from such corpera include one or more entities per sentence. 
In our attempt to recognize polymer names in full text documents, we face a very imbalanced dataset where most sentences do not contain a target entity as there are only a handful of target entities per document.
%(we found less 2\% of words in sample publications are polymers in our test dataset)
While there has been much interest in machine learned recognition of biochemical entities~\cite{jessop2011oscar4,rocktaschel2012chemspot,leaman2015tmchem,swain2016chemdataextractor}, there has concurrently been significant effort involved in selecting and (often manually) generating quality data for
trainable statistical NER systems~\cite{krallinger2015chemdner}. 
Previous work has also found that even state-of-the-art NER systems do
not typically perform well when applied to different domains~\cite{krallinger2013overview}. 
Therefore the problem of training machine learning models for recognizing new scientific entities remains a challenging one.

Due to these issues, identifying polymeric names is a non-trivial exercise not only for computers, but also for experts within the field. 
We expect such a scenario that is not unique to polymer science.
Our long-term goal is to build a hybrid human-computer system in which we leverage both 
human and machine capabilities to extract polymer names and their properties. 
In this work, we focus on the task of identifying polymer names.

%------------------------------------------------------------------------------

%------------------------------------------------------------------------------
\section{Related Work}
\label{sect:relatedwork}
NER and other information extraction tasks rely on large amount of training data, which is expensive to obtain.
Weakly supervised learning methods work with much less training data and aim to address this challenge.
They generally fall under two categories: semi-supervised learning and active learning~\cite{zhou2017brief}.
Semi-supervised learning attemps to automatically label data based on a set of labeled data.
Active learning assumes there is a source of knowledge, such as a human expert, that can be queried to label a selected batch of unlabeled data. 

\subsection{Semi-supervised Approaches}
\textit{Bootstrapping} is a semi-supervised technique, which starts from a small set of seed relation instances and iteratively learns more relation instances and extraction patterns.
Snowball~\cite{agichtein2000snowball} which improved the DIPRE system~\cite{brin1998extracting}, used an intuitive idea to collect new entity relations using a set of seed entity pairs.
The authors start with a handfull of seed tuples of \textit{ORGANIZATION} and \textit{LOCATION}, Snowball attempts to learn the relation \textit{HeadquarteredIn} by assuming that every time the tuples appear in close proximity to each other, the text in between illustrates the desired relation.
This text can then be used to discover new tuples, which can in turn be used as seed for the next discovery iteration.
Of course, organizations may be located but not headquartered in multiple cities, hence it is important to inspect the quality of extraction patters to reduce noise in the generated output.

\textit{Distant supervision} maps known entities and relations from a structured knowledge 
base onto unstructured text~\cite{peters2014machine,zhang2013geodeepdive}. 
With freely available structured knowledge base such as DBPedia~\cite{auer2007dbpedia} and Freebase~\cite{bollacker2008freebase}, it is possible to leverage a large set of known entity pairs to generate training data
%~cite{mintz2009distant}; 
%in this work authors assume that if two entities participate in a relation, any sentence that contains both entities descripes that relation. 
%Because that is not always the case, they extract features from different sentences to define
%lexical, syntactic and named entity tag features. 
%They use standard multi-class logistic regression as the classification algorithm and reach almost 70\% of precision based on human judgment.
For the last decade, probabilistic approaches have been proposed to allow the system to select automatically
For example PaleoDeepDive~\cite{peters2014machine}, built upon DeepDive~\cite{zhang2015deepdive}, automatically extracts
paleontological data from text, tables, and figures in scientific publications. 
For good performance in such applications, such approach often relies on and extends large databases: for example,
PaleoDeepDive uses PaleoDB~\cite{PaleoDB}. 
The system labels any entity pair that appears in the database as \textit{True}.
The user defines features (e.g. entity pairs that are too far apart are marked \textit{False}), the system then uses statistical inference to determine the probability that each newly discovered pair of interest is \textit{True}.

Similarly \textit{Data programming} uses
\textit{labeling functions} (user-defined programs that provide labels for subsets of data)~\cite{ratner2016data}. 
Errors due to differences in accuracy and conflicts between labeling functions are 
addressed by learning and modeling the accuracies of the labeling functions. 
Under certain conditions, data programming achieves results on par with those of supervised learning methods.
While writing concise scripts to define rules may seem to be a more reasonable task for annotators 
than exhaustively annotating text, it still requires expert guidance.  
In data programming as in boostrapping and distant supervision it is important to evaluate the quality of functions and extraction patters to decrease noisy patterns.

\subsection{Active Learning}
Active learning~\cite{zhou2017brief} assumes that the gold standard labels of unlabeled instances can be queried from an oracle (domain expert or source of knowledge). 
The goal of active learning is to decrease the cost of labeling by requesting a limited number of labels from the oracle, that have been deemed most valuable by the learner.
Uncertainty sampling approaches define ``valuable'' data by measuring uncertainty in the predictions.
For example, in the case of a single learner, querying predictions with maximum entropy in which the learner assigns all classes with equal probability~\cite{lewis1994heterogeneous} or predictions closest to the decision boundary in the case of support vector machine classifiers~\cite{campbell2000query}.
In the case of multiple learners, query-by-committee requests labels for unlabeled instances on which the learners disagree the most~\cite{seung1992query}.
Uncertainty sampling and query-by-committee are representative approaches based on informativeness, where informativeness measure show well an unlabeled instance helps reduce the uncertainty. 
Another selection criterion addresses representativeness, which measures how well an instance helps represent the structure of input patterns; in this case selection is made by querying data from unlabeled clusters of data~\cite{nguyen2004active,dasgupta2008hierarchical}.

\subsection{PolyNER}
In this work, we implement a low-cost scientific fact extraction system. 
We ask: how can we quickly generate annotated data for scientific named entity recognition?
We face two challenges: (1) lack of (expensive) training data in some fields, including our own polymer science application which lacks free access to large polymer databases; and (2) the need for domain expert curators, which impedes the use of crowdsourcing platforms such as Amazon Turk~\cite{buhrmester2011amazon} or Figure8~(\url{https://www.figure-eight.com/}).
For these reasons, our approach is similar to bootstrapping but applied to named entities rather than entity pairs. 
We generate approximate labels by extracting targets similar to a few seed entities using word representations and vector similarity measures.
To address noise in the output, expert label small batches of candidates, which can be used to train a word vector classifier and propose final scientific named entities.
Eventually, we will explore using polyNER along with distant supervision and data programming to extract polymer properties.
An example rule could be defined as: \textit{if a sentence contains a polymer name and the words ``glass transition'', then extract number(s) in the sentence as potential glass transition temperature(s) for that polymer.}
We discuss the architecture of our system in more details in the next section.




 



%------------------------------------------------------------------------------

%------------------------------------------------------------------------------
\section{Design and Implementation}
\label{sect:apner_architecture}
As previously mentioned, our main goal is designing polyNER is to slash labeling costs by reducing the time and effort spent by experts to generate training data. 
Rather than labeling entire documents and phrases, annotators label proposed candidate entities to be classified.
%\logan{What is downstream in this metaphor? I also don't think it is a needed detail}\roselyne{ok, removed, it stemmed from the idea in my head about mentioning that our labels can be used by Hong, and others later, picked up from snuba, but probably can be emphasized better or elsewhere}.
Earlier results show that with two hours of labeling we can achieve
precision or recall (but not yet both) on par with state-of-the-art domain specific software, by selecting an ensemble of classifiers for discrimination~\cite{tchoua2019polyner}. %\loganfussingaboutrecallandprecision \roselyne{noted, will specify in results that we also use ROC and PR curves}

Here, we refine polyNER components and incorporate active learning with different sampling strategies in order to further improve performance.
PolyNER uses word representations and minimal domain knowledge (a few
seed entities) to produce a small set of candidates for expert labeling;
labeled candidates are then used to train named entity word vector classifiers.
We integrate an active learning loop into polyNER's architecture to incrementally improve classifier performance.

In order to explore whether the use of word vector coordinates as features can accelerate the learning process,
we define and compare three alternative sampling strategies: a random strategy that we use as a baseline,
and two NLP-based filtering methods. 
%\ian{Are the three strategies all an integral part of polyNER, or are they alternatives that we compare? Unclear. I reworded to say they are alternatives, is that right?}\roselyne{Yes to alternatives that we compare.}
We also apply these methods against two different candidate pools, 
one set of unlabeled nouns and another set of approximately labeled nouns deemed \textit{similar} to commonly used known entities from our corpus.
%\ian{The candidate pool reference is confusing to me because in the text that follows you seem to refer to just one
%pool, the ``NLP-filtered candidates."}\roselyne{I specifically describe two pools, the NLP filtered, and the distance NLP-filtered, perhaps too wordy?}
We describe our sampling strategies and approximate labeling in more details in this section.
% for maximum entropy uncertainty sampling, which we describe in this section.
%\logan{We have mentioned these two different pools twice now, but haven't given a hint as to what they are. I think we should describe them succinctly here.}\roselyne{Done I think}
The general architecture of polyNER is illustrated in Figure~\ref{fig:architecture}.
We also describe the labeling process, and the training and testing configuration for our word vector classifiers in the active learning loop. 

\begin{figure*}[!t]
{\includegraphics[width=\textwidth]{figures/architecture.pdf}}
\caption{\label{fig:architecture} PolyNER system showing the different phases of polyNER including the NLP-filtering step, the initial sampling and the newly integrated active learning loop to classify scientific named entities. 
%\ian{(1) It is really important to be consistent in how your refer to things. Here, the figure says ``Bootstrapping and Initial Labeling" for (1), but the caption says ``initial sampling and labeling.'' Pick one and use it consistently. (2) You are inconsistent in capitalization. (3) The caption mostly repeats the labels in the figure, which seems redundant.
%It could be higher level.}
	%\logan{Should we differentiate between the "initial sampling" strategy done to bootstrap our training set and the "active learning sampling" strategy used to identify the next things to label? Mixing the two up is going to confuse people}\roselyne{OK will adapt the rest of the text and section titles too.}
}
\end{figure*}

\subsection{Computing Word Embeddings}\label{sec:we}

A word embedding method 
%\ian{You referred to ``vector similarity" but then switch to talking about word embedding.
%Maybe say ``We use word embedding methods to obtain the vectors that we use for determining similarity.''} 
maps each word
in a document to a vector in an n-dimensional real vector space that
represents the linguistic context in which the word appears. (This mapping may
be based, for example, on co-occurrence frequencies of words.) 
We can then
determine the similarity between two words by computing the distance between
their corresponding vectors in the feature space.

We use Word2Vec, a recent, light-weight and easy-to-use implementation of context-based vector representations~\cite{mikolov2013efficient,mikolov2013distributed}.
Specifically we use the Gensim continuous bag-of-words
(CBOW) implementation of the Word2Vec
algorithm~\cite{rehurek2010software} to generate vectors.
We describe in Section~\ref{sec:wes} how we select appropriate values for the
Word2Vec (vector) \texttt{size} and \texttt{window_size} parameters

\subsection{NLP-Filtering}\label{sec:filter}
%\subsubsection{Word Embedding Model}
%\logan{Why is this titled "Word Embedding Model" and you start by describing a classifier?}
%We hypothesize that we can implement a classifier, which can detect word vectors for polymers based on their context. 
%\logan{This is the only section that you start with a "hypothesis"}\roselyne{ok removing this and moving section}
%\logan{Where is this section in F\ref{fig:architecture}?}
%We generate an unsupervised word embedding model using out entire corpus and train classifiers on vector representations using labels generated via the active learning(step 3 of Figure~\ref{fig:current}).
%Finally, in step 4, we test our classifiers against all the NLP-filtered words from the test corpus, as shown on Figure~\ref{fig:current}.

%\logan{Make section labels line up with the names in the figures}
First, we define an
NLP filtering preprocessing step to filter out
string
%\ian{Maybe the wrong term, as the first example is ``polyethylene;'', which is not a ``word'' but a string?}
 in scientific publications that are unlikely to be polymer referents. 
%In other words, we remove obvious non-candidates.
First, we remove
unwanted characters (e.g. `:', `.', `,', `:', `-') from the beginning and the end of each
candidate and eliminate numbers (including numbers followed by common units).
This step allows us to recognize, for example, ``polyethylene;'' and eliminate it as yet another unique (erroneous) candidate. 
%\ian{I can't tell what you are saying in the preceding. Can we simply say, ``thus, for example,
%we can eliminate ``polyethylene;'' as a string distinct from ``polyethylene.'' [I am not sure if I have worded this
%right.]}\roselyne{not sure I understand that better, but I think I can rephrase}
Hypothesizing that names of scientific entities will not, in general, be English
vocabulary words, we remove words found in the SpaCy dictionaries
of commonly used English words~\cite{choi2015depends}. (We manually remove common polymer
names, such as polystyrene and polyethylene, from the dictionaries.) 
We also use
SpaCy's part-of-speech tagging functionality to remove non-nouns.  
Finally, we remove plurals (e.g.,
polyamides, polynorbornenes), as they can represent polymer family names.
Note that these steps are generalizable and applicable to multiple science fields.
%\logan{2 is the only number that does not start with "We remove." Should you make this a bulleted list?}\roselyne{Good point, but then this is hardly a section I think, I will try and see how it looks. Also some of the bullet text is short and some is long. e.g. numbers vs, entire sentence.}
We refer to the set of words that result when these filtering operations are applied to our corpus (the output of step 1 in Figure~\ref{fig:architecture}) as the \emph{NLP-filtered candidates}.


\subsection{Initial Labeling}
While the previous step %\ian{Should this be singular?}
reduces the numbers of candidates and the imbalance of the dataset (target vs.\ non-target entity ratio), 
there still remains a large pool of potential candidates from which entities are to be selected.
In our experience, only 5\% of NLP-filtered words are polymer names.
%\ian{I don't think that a ratio can be a percentage. You may mean ``only 5\% of NLP-filtered words are polymer names.}
In order to avoid presenting experts with mostly negative examples, hindering meaningful classification,
we attempt to boost the number of polymer entities in the first batch of candidates to be annotated by using word vector distance metrics between candidates and a set of \emph{seed entities}:
words that are observed to occur frequently  
in a subset of publications or that are suggested by experts.
We discuss this distance metric in more details below.
%Candidate batch sizes are determined based on labeling time estimates and availability of expert curators.
Based on preliminary experiments, we set the size of each batch of strings to be labeled to 200, 
or about an hour of expert time.

\subsection{Sampling Strategies}
%While the previous steps reduces the numbers of candidates and the imbalance of the dataset (target vs. non-target entity ration), there still remains a relatively large pool of potential candidates to select entities from.
%In order to achieve higher classification accuracy\textemdash
%by decreasing the number of potential false positives (candidates incorrectly identified as targets by our classifier)
%\textemdash we want to carefully select examples to be labeled by experts.
%\logan{Why is labeling "false positives" bad?}\roselyne{It just takes more time from the expert, also we can't give the expert only negative examples.}
We implement three sampling strategies, which we refer to as \textit{Random}, \textit{Uncertainty-Based Sampling (UBS)} and \textit{Distance Uncertainty-Based Sampling (Distance UBS)}.
We apply each of these strategies to our NLP-filtered candidates to determine which candidates to label.
%Based on preliminary experiments, we set the size of batches of strings to be labeled to 200 or about an hour of expert time.
%\logan{As my comment in F\ref{fig:architecture}, we have two different kinds of sampling strategies and I think we should describe them separately. What do you think?}\roselyne{I'm separating the initial sampling and adding it to the figure}

\subsubsection{Random Strategy using NLP-Filtered Candidates}
In the first strategy, we randomly select 200 out of the pool of unlabeled NLP-filtered candidates from the entire corpus of publications.
Note that the imbalance between polymers and other NLP-filtered \textit{tokens} (words or space separated strings)  is still significant (less than 5\% in our experience).
%We exclude tokens included in our test set.\ian{I have no idea what the preceding text is saying. What does it mean to have an imbalance with things that do not exist?
%Could this paragraph simply be, ``As a baseline, we will present results for a random selection strategy,
%in which 200 candidates are selected at random from the NLP-filtered candidates.''}\roselyne{It's missing dictionary, that do not exist in the dictionary}
%\ian{Separate issue: as far as I know, you have not introduced the concept of a test set, so the reference to test set is confusing.}

\subsubsection{Uncertainty-Based Sampling using NLP-Filtered Candidates (UBS)}
Our second strategy applies maximum entropy sampling to the NLP-filtered candidates. % from the full corpus of documents used in the random strategy.
%\logan{Same pool}
As previously mentioned, maximum entropy selection is an uncertainty sampling method that
identifies data points for which a classifier predicts outcomes that lie near the decision boundary 
between classes. 

%\ian{I don't understand the next text. Could we write, 
%``For example, when predicting whether or not a word vector represents a polymer, 
%maximum entropy arises when the classifier assigns equal probability to the polymer and not-polymer cases.''}
For example, when predicting whether or not a word vector represents a polymer, maximum entropy arises when the classifier assigns equal probability to the polymer and not-polymer cases.
As we have two classes, this equal probability is 0.5.
%For example, in our case, when predicting whether or not a word vector represents a polymer, 
%the classifier assigns equal probability to either case.
%\ian{Can we say, ``As we have two classes" rather than ``in the binary case"?}
%In the binary case, probabilities range from 0 to 0.5. 
%\ian{Why can't you have a probability of 1, for say ``polystyrene''?}\roselyne{You're right, you can have that, and the text is confusing, if you you have a probability of 1, then |1-0.5| gives you 0.5 and this will not be considered as a confusing example, the examples with probability p with |0.5-p| close to 0 are the ones considered confusing} 
Therefore, we predict outcome for all our NLP-filtered candidates and obtains a probability $p$ for each data point. We compute a list of $0.5-p$ values for all unlabeled data and sort the list in ascending order.
Points with scores closest to $0$ are most uncertain.
We select the first 200 entries from this ordered list to be labeled by experts.

\subsubsection{Uncertainty-Based Sampling using NLP-Filtered Distance Candidates (Distance UBS)}\label{sec:we}
Our second strategy selects NLP-filtered candidates that are used in similar contexts to known entities..
The intuition here is that a candidate is more likely to be a target referent (a name, acronym, synonym, etc.) if it used in a similar context.
For example, the polymer name ``polystyrene'' in a sentence ``The
melting point of polystyrene is ...'' suggests that X may also be a polymer in the
sentence ``The melting point of X is ...''.

We use the word embeddings introduced in Section~\ref{sec:wes} to capture this notion of context,
and vector distance between word vectors as a measure of similarity.
Whether or not this approach works in practice will depend on whether 
polymer names are in fact used in consistent contexts that is captured by our 
word embedding  vectors. 

We can then determine, for each NLP-filtered word, the extent to which it occurs
in a similar context to the representative\ian{Your referred above to ``seed entities.'' Now you switch to ``representative polymers'' and ``representative entities.'' I recommend sticking to one term and using it consistently.} polymers, by computing the similarities
between the word's vector and those for our known representative entities. 
We experiment with one and more seed entities; when dealing with multiple representative entities,
we use the lowest distance score for ranking candidates.
%Here too, we exclude terms that exist in our test set of documents.

\begin{figure}
\centering
%\scalebox{0.4}{\includegraphics{figures/al_setup.pdf}}
\includegraphics[trim=0.15in 0.1in 0in 0.in,clip,width=3.5in]{figures/al_setup.pdf}
\caption{\label{fig:current} The active learning experiment set up; 1) we first generate an unsupervised word embedding model using our entire corpus, 2) we propose NLP-filtered candidate entities to untrained and expert annotators before classifying their word vectors in 3). We select the best-performing classifier for uncertainty-based sampling of labels for the next round of active learning. In 4), we evaluate this word vector classifier on all NLP-filtered words from a set of test documents. 
%\logan{What are the blue docs? Why don't they have an arrow? Where do the NLP candidates for 2 come from?}
%\roselyne{They come from the entire pool of NLP-filtered candidates (excluding the test set, will make sure this is clear}
%\ian{Could this fit in a single column?}
%\ian{Here and in other figure, I find it confusing to have the description precede the number.}
}
\end{figure}

\subsection{Active Learning Loop}
As we have no prior knowledge of the distribution of target entities in the vector space, 
we use multiple classifiers at each iteration of the active learning process. 
%\ian{What does it mean to ``save''?}\roselyne{we use,keep, mark? the classifier that will be used for sampling.}
We use the best performing classifier on the labeled candidates for subsequent maximum entropy-based uncertainty sampling. 
%\logan{You have neither defined nor motivated MEU, I think}\roselyne{See if you would add something to the previous strategies (UBS). That's where I've tried to define it, tell me if it's not clear}.
The requested labels are annotated by humans to serve as addition training data for the next learning iteration. 
We describes these steps in more details in the following sections.

\subsubsection{Untrained and Expert Labeling}
As explained in section~\ref{sect:background}, recognizing polymers can require more or less domain expertise.
We assign two domain experts to annotated candidates generated using our two maximum entropy-based uncertainty sampling (\textit{UBS} and \textit{Distance UBS}). 
Each expert annotates one strategy but we perform crosschecking for 10\% of the first batch of labels to get a measure of agreement between experts. 
%We confirm agreement between labels for all but 1 of the set of 20 candidates or an agreement of 95\%.
%\logan{The inter-relater score is a result, not part of the architecture}\roselyne{ok, don't remove, come back to this to make sure I add it later}
Experts simply approve or reject candidates using the web interface shown in Figure~\ref{fig:polyner},
a task that is far more efficient than reading and annotating words in text.
The interface
provides example sentences as context for ambiguous candidates,
and allows the expert to access the publication(s) in which a particular candidate
appears. % when desired.

We aim to reduce the amount of costly expert time used to obtain labels.
%Expert time is costly and we aim to reduce the cost of obtaining labels.
Therefore for our baseline of randomly sampled NLP-filtered nouns, we experiment with a two-phase review process.
Tokenization is one of the largest sources of error for scientific entities such as polymers, 
which contain characters such as `:', `(',
'\textendash', `,' etc. 
Tokenization can also generate incoherent tokens from text, equations, captions, etc.
Such obvious non-candidates can be fairly easily detected by non experts.
For example, an untrained human annotator may be able to recognize that `$d\Sigma/d\Omega)(Q$' is not a polymer name, and thus save time for the experts.
Hence, we assigned two graduate student labelers to curate the candidates generated by the random sampling strategy, which are less likely to contain target entities.
We asked these untrained labelers to reject obvious non-candidates via the previously mentioned web interface. 
Our expert polymer scientists then reviewed the remaining
candidates, indicating for each whether it is in fact a polymer referent. % and submits a final review.

%While we first used this two-phase review process for the random strategy, 
%we envision generalizing and leveraging humans with different expertise through multi-phase reviews for all strategies to further save costs.


\begin{figure}
\centering
\frame{\includegraphics[trim=0in 0.1in 0.1in 0.in,clip,width=3.5in]{figures/expert_labeling.png}}
\caption{\label{fig:polyner} Web interface for expert review of candidates.
The expert indicates whether the name (column 1) is a polymer (tickbox in column 2), 
providing notes if desired (column 3). 
Clicking on ``?'' delivers up to 25 more example sentences.
}
\end{figure}

\subsubsection{Classification or Candidate Discrimination}
We use multiple classifiers that we concurrently trained and test on the same data in steps 1, 2, and 3 of Figure~\ref{fig:current}.
The classifiers include the scikit-Learn~\cite{scikit-learn} implementations of Decision Tree (DT), Gradient Boosting (GB), K-Nearest Neighbor (KNN), Logistic Regression (LR), Linear Support Vector Machine (SVM), Naive Bayes (NB), and Random Forest (RF). 
%\logan{Something to consider for future: I've been nervous a long time about how we use so many models. How good is your grid search CV for these models? Many (KNN, SVM) are super sensitive to hyperparameters, and I'm worried we are testing 7 models badly and that we'd have better results with tuning 1 really well}
%\roselyne{Yes, this is definitely a lesson taken. I was just telling Kyle, I know exactly how I would change this experiment and others based on my experience now. Perhaps I need to say that in the discussion!}\roselyne{Added note to include in discussion}
Our goal here is to explore the word embedding space and determine which classifier(s) works best for detecting our scientific named entities.
As previously mentioned, we select%\ian{that word save again.} 
the \textit{best-performing} classifier on labeled candidates for subsequent uncertainty sampling.
When defining best performance, we prioritize recall, or retrieving a maximum of targets, over precise extraction.
In other words, extracting a higher number of targets potentially requiring additional curation is favored over obtaining fewer correct targets.
%\logan{Can you express this quantitatively?}\roselyne{I have favored recall, maybe I should just say that? better?}
In each case, we use the word embedding for each string as input features.
%\logan{Do you mean "use the word embedding for each word as input"? All word vectors should have the same dimensions}\roselyne{yes, they do, I want to indicate exactly that the valye of the dimensions word vector itself is used as features, I wasn't sure if the embedding for each word as input was clear enough.}





%------------------------------------------------------------------------------

%------------------------------------------------------------------------------
%\section{Parameter Study}
\label{sect:parameter_study}

%------------------------------------------------------------------------------

%------------------------------------------------------------------------------
\section{Evaluation}
\label{sect:apner_results}
We first report on a study in which we evaluate the generation of distance candidates using the vector similarity measures and seed entities. 
\logan{Would using "candidate entities" rather than "distance candidates" make more sense?}
\logan{If F\ref{fig:current} describes your setup. Why not reference it here?}
We then discuss the results of four rounds of active learning using multiple word vector classifiers and our three sampling strategies: random, UBS, and Distance UBS.
Finally, we experiment with word representations enhanced with character-level information using FastText~\cite{bojanowski2016enriching,joulin2016bag}.

\subsection{Dataset}
We work with a corpus composed of \nistnum{1690} full-text publications in HTML format downloaded from \textit{Macromolecules}, a relevant journal in polymer science.
These documents comprise \nistnum{381947} sentences and \nistnum{9229417} (\nistnum{253195} unique words or ``tokens'').
We set aside a test set of  100 documents with  \nistnum{22664} sentences and \nistnum{508391} (\nistnum{36293} unique) tokens. 
We engaged six experts to identify one-word polymer names from our test set.
They extracted 467 unique one-word polymer names, which we use as gold standard.
When testing against this gold standard, we evaluate using a total of 9656 NLP-filtered nouns from the 100 documents.
We evaluate extraction accuracy in terms of precision and recall.
Recall refers to the fraction of actual positives that
are labeled correctly and precision to the fraction of predicted
positives that labeled correctly.
\logan{What part of this data can we release, if any?}

\subsection{Word Embedding Settings}
\logan{Needs better topic sentence. Say why you are doing this more clearly, less about the "what"}
We explore how the choice of seed entities, the internal parameters \textit{vector} and \textit{window_size}  impact the number of target entities retrieved using similarity measures.
In order to estimate the entity-richness of our large pool of distance candidates and since we do not have manually extracted data for the entire corpus, we create a list of \nistnum{10000} distance candidate vectors most similar to our seed entity vectors and report the fraction of gold standard polymer extracted. 
Based on our experience of $\sim$ 5 polymers per document (\nistnum{8450} polymers for \nistnum{1690} documents), we can expect a fraction of the polymers found in the 100 gold standard documents to be extracted in the \nistnum{10000} candidates most similar to our seed entities.
\logan{This is confusing. Why are we using a model to estimate the total population of the polymers and not just assuming the gold-standard is a representative sample?}
We evaluate the entity-richness of polymers in this pool of candidates by measuring the percentage of the 467 manually extracted polymer names that it yields;
we use lower-case exact string matching between the gold standard polymer names and the proposed distance candidate strings to determine if a candidate is a polymer.

\subsubsection{Seed Candidates}
Here, we focus on how the choice of seed entities affects recall, which we take as a measure of the entity-richness of the pool of candidates.
\logan{This is confusing. Is the fraction of polymers in the candidate pool not the recall?}
In previous work using the same corpus~\cite{tchoua2016hybrid,tchoua2016hybridi}, we build a dictionary of polymer names using a rule-based approach and aggregating synonyms across ChemDataExtractor records\textemdash a record consists of all information found about a chemical entity in a document.  
Having built this dictionary, we can identify the 10 most frequently occurring polymers in our corpus and their acronyms.
We assume that frequent polymers provide a large number of example sentences that illustrate context in which polymers are commonly used.
\logan{Do you think it would be better to use the gold standard, where we have certainty of the labels, to estimate the most-frequent polymers?}
Hence, we first test the most common, the three most common and the ten most common polymers as seed entities.
\logan{The circularness of this strategy makes me uncomfortable. We use an NER model to determine the initial training set for the NER model? How would you address this criticism?}
We experiment with including and excluding their acronyms as corresponding additional seed entities.
(Note that this modest set of 1, 3 and 10 seed entities could also be suggested by an expert.)
Table~\ref{tab:candidate_generation} shows the results for this set of experiments on rows one through six.
When using \textit{polystyrene} (the most commonly used name) as a seed entity, the pool of candidates contained 33.55 \% of the 467 gold standard polymers.
\logan{Another potential criticism: Would I need a gold standard to build the NER method? I thought your method is supposed to help me avoid having to do the work for generating a gold standard?}
We note a 2 \% increase in the fraction of polymers retrieved when using \textit{polystyrene} and its acronym compared to using \textit{polystyrene} alone (37.69 \%).
\logan{Is this an important detail? Why does it need to stay [I think this section is a little dense]}
The fraction of polymers increases by 10 \% when using the three most frequent polymers as seed entities (from 33.55 \% to 46.9 \% and 47.97 \% with acronyms); the three most frequent polymers in our datasets are \textit{polystyrene},\textit{poly(methyl methacrylate)},\textit{polyethylene} and their acronyms (\textit{PS}, \textit{PMMA} and \textit{PE})
Using ten instead of three entities however, only slightly increases the yield of polymers by less than 1 \%, from 47.97 \% for three frequent polymers with acronyms to 48.39 \% for ten frequent polymers with acronyms.

\begin{table}[ht!]
\centering
\caption{Fraction of gold standard polymer names extracted from pool of \nistnum{10000} \textit{distance} candidates using different candidate generation methods.\label{tab:candidate_generation}
}
\vspace{2ex}
%[35.55, 37.69, 46.9, 47.97, 46.47, 48.39, 46.68, 36.4]
\setlength\tabcolsep{3pt}
\begin{tabular}{|C{0.1in}|C{2.4in}|C{0.6in}|}
 \hline
\textbf{\#} & \textbf{Candidate Generation Method} & \textbf{Fraction of polymers extracted}  \\
\end{tabular}
\begin{tabular}{|C{0.1in}|R{2.4in}|L{0.6in}|}
\hline
 1 &    Polystyrene & 35.55\%  \\
\hline
 2 &    Polystyrene with acronym (PS) & 37.69\%\\
\hline
 3 &    3 most frequent polymer names & 46.90\%\\
\hline
 4 &    3 most frequent polymer names with acronyms &  47.97\%\\
\hline
 5 &    10 most frequent polymer names & 46.47\%\\
\hline
 6 &    3 most frequent polymer names with acronyms & 48.39\%\\
\hline
 7 &    $\chi$DB polymer names & 46.68\%\\
\hline
  8 &  crowDB polymer names    & 36.40\%\\
\hline
\end{tabular}
\end{table}

In a second set of experiments, we explore the idea of using larger numbers of seed entities to increase the fraction of polymers retrieved in the pool of candidates.
We have built a small database of polymer properties ($\chi$DB) in previous work~\cite{tchoua2016hybrid,tchoua2016hybridi}. 
Our corpus of \nistnum{1690} publications included 111 our of 175 $\chi$DB polymers.  
We also scraped CrowDB, which lists some polymers and their properties at \url{http://polymerdatabase.com/} for polymer names.
In this case, 32 out of 295 scraped polymer names were found in our corpus.
We experiment using these 111 and 32 seed entities to extract polymers. The fraction of polymers from our gold standard is shown on rows seven and eight in Table~\ref{tab:candidate_generation}.
These results confirm using more entities does not increase the yield of polymers, as some polymers have low frequency in the corpus, words that are most similar are less likely to be targets.
For the remainder of the experiments, we use the three most frequent polymers and their acronyms. 
\logan{I think the metric we use for choosing an algorithm here (fraction of entries in the training set) is too indirect. Why not use the performance of the initial ML model to rate how good the training set is? A hypothetical concern with the metric as well: Does your current metric mean a pool with 100\% polymers would be the best? Wouldn't that lead to the same problem of imbalance as having no polymers?}


%\kyle{perhaps better as a table. The long labels are hard to read}\roselyne{ok}
%\begin{figure*}
%\centering
%\begin{minipage}[b]{.4\textwidth}
%\includegraphics[trim=0in 0.1in 0.1in 0.in,clip,width=1.0\textwidth]{figures/candidate_generation_method1.png}
%\caption{\label{fig:cand_generation1} Fraction of polymer retrieved for various candidate generation methods using most common, three most common and ten most common polymer names as seed entities.
%}
%\end{minipage}\qquad
%\begin{minipage}[b]{.4\textwidth}
%\includegraphics[trim=0in 0.1in 0.1in 0.in,clip,width=1.0\textwidth]{figures/candidate_generation_method2.png}
%\caption{\label{fig:cand_generation2} Fraction of polymer retrieved for various candidate generation methods using three most common (with and without acronymss), $\chi$DB and CrowDB polymer names as seed entities.  
%%\kyle{Not sure this needs to be a separate graph?}
%}
%\end{minipage}
%\end{figure*}
%Together
%[35.55, 37.69, 46.9, 47.97, 46.47, 48.39, 46.68, 36.4]
%[35.55, 37.69, 46.9, 47.97, 46.47, 48.39]
%\begin{figure}[!t]
%\centering
%\includegraphics[trim=0in 0.1in 0.1in 0.in,clip,width=3.5in]{figures/candidate_generation_method1.png}
%\caption{\label{fig:cand_generation1} First set of experiments with seed entities showing noticeable improvement from 1 to 3 and less improvement from 3 to 10 seed entities.
%}
%\end{figure}
%[46.9, 47.97, 46.68, 36.4]
%\begin{figure}
%\centering
%\includegraphics[trim=0in 0.1in 0.1in 0.in,clip,width=3.5in]{figures/candidate_generation_method2.png}
%\caption{\label{fig:cand_generation2} First set of experiments with seed entities showing that using more polymers as seed entities does not necessarily enrich the pool of distance candidates.
%}
%\end{figure}

\subsubsection{Word Embedding Window and Size Parameters}
We measure the impact of the \textit{window} and \textit{size} on the fraction of polymers extracted from the gold standard in the list of \nistnum{10000} candidates. \logan{Where are the 10k candidates from again? Could we reference this set by its purpose rather than its size?}
\logan{Need a reason why this is important? Also, this statement again raises my concern about using "fraction extracted from gold standard" as a metric for quality of the initial training set?}
The \textit{window} represents the maximum distance between the current and predicted word within a sentence. In other words, it represents the number of words before and after each word considered by the neural network to generate a vector representation for that word. 
For each parameter setting, we measure the yield of polymers ten times for each window and vector size setting.
We observe slightly higher fraction of polymers retrieved for window sizes of 1 and 2. The yield subsequently decreases and more noticeably with window size larger than 5 (see Figure~\ref{fig:window_size}).
The \textit{size} determines the size of the word vector (features for our classifier).
We varied this parameter between 100 and 500.
The average yield of polymer names was measured at 49.3 with a standard deviation of less than 1\% (see Figure~\ref{fig:vector_size}).
%[49.67880085653105, 49.25053533190578, 49.46466809421842, 49.46466809421842, 48.82226980728051]
\logan{Need a concluding sentence?}


\begin{figure*}
\centering
\begin{minipage}[b]{.4\textwidth}
\includegraphics[trim=0in 0.1in 0.1in 0.in,clip,width=1.0\textwidth]{figures/window_size.png}
\caption{Impact of varying window size on fraction of polymers retrieved.}\label{fig:window_size}
\end{minipage}\qquad
\begin{minipage}[b]{.4\textwidth}
\includegraphics[trim=0in 0.1in 0.1in 0.in,clip,width=1.0\textwidth]{figures/vector_size.png}
\caption{Impact of varying vector size on fraction of polymers retrieved. \logan{Can we combine this with F\ref{fig:window_size}? (If we keep it) }}\label{fig:vector_size}
\end{minipage}
\end{figure*}
 %\kyle{Not sure this fig is worth including}\roselyne{Agreed that it's not so interesting but i think it goes with the windows one and i will combine the two figures you think should be together.}\roselyne{May remove as paper gets longer}

\subsubsection{Initial Classifier}
Having explored the parameters for distance candidate generation, we generate a pool of \nistnum{10000} candidates most similar to the three most frequent polymers and their acronyms from a test corpus of \nistnum{1690}, excluding any token found in our test documents. 
\logan{Is the 10k entries mentioned in the previous section this set? If so, we need to revise the order of these sections?}
The candidates are arranged in decreasing order of similarity scores and we use a window size of 2 and a vector size of 200. 

In order to balance the initial training set (increase the potential number of target entities),
we initially train the classifier using the first
200 entities from the above mentioned list of \nistnum{10000}.
Recall that the batch size of 200 was set based on an estimate of 30-60 minutes of expert time.
\logan{Maybe we can be specific here about how much time it took. "We found that a batch of 200 required ..."}
As described on Figure~\ref{fig:current}, in the first step, we train and test on candidates generated from a word embedding model generated using only the training documents. 

Figure~\ref{fig:roc_init} shows the Receiver Operating Curve (ROC) for the initial classifier with highest recall when evaluating on the test corpus.
\logan{I do not like this as a topic sentence because it doesn't offer any indication of what to draw from the figure. I prefer ones that are "We found that X, as shown in Figure Y." Topic sentences with some editorial comments worked in help keep the narrative of the purpose of your study alive.}
The ROC plot is an evaluation measure that is based on two basic evaluation measures: specificity (true positive rate) and sensitivity (true negative rate).
Sensitivity is the same as recall. Specificity is a measure of the true negative rate (the proportion of actual negatives that are correctly identified as such).
A classifier with the random performance level always has the same 0.5 true positive rate and false negative rate.
\logan{Not true. The TPR depends on how many points you sample, and is always equal to the FPR}
A classifier with the perfect performance level shows a combination of two straight lines immediately showing a true positive rate of 1.0 and remaining there as recall increases.
Classifiers with meaningful performance levels usually lie in the area between the random ROC curve (baseline) and the perfect ROC curve.
\logan{Also not quite correct. All classifiers with better-than-random performance lie between random and perfect. Maybe just skip the notion of betweenness and talk about the AUC}
The area under the curve (AUC) measures the area under the ROC.
Ideally the AUC of a learning algorithm is above 0.5. 
Since AUC takes into account true negatives or correctly predicted non-polymers and our dataset is imbalanced containing more non-polymers than polymers, we also plot the Precision Recall Curve (PRC) to visualize the tradeoff between precision and recall.\logan{A little wordy, can you streamline this sentence?}
\logan{This paragraph contains no conclusions, it just explains the ROC. Add something about the results.}

While the AUC for the initial base KNN classifier is above random performance in Figure~\ref{fig:roc_init}, the initial PRC shows poor precision regardless of recall (see Figure~\ref{fig:prc_init}).
\logan{Better topic sentence. It provides a hit of what to look for. But, you do not follow it up with any details supporting this claim}

\logan{This is a long section for just describing how you made the classifier and that it is pretty bad. Do we need this section?}

\begin{figure*}
\centering
\begin{minipage}[b]{.4\textwidth}
\includegraphics[trim=0in 0.1in 0.1in 0.in,clip,width=1.0\textwidth]{figures/roc_init.png}
\caption{Receiver Operating Curve for KNN model for initial round of labels.}\label{fig:roc_init}
\end{minipage}\qquad
\begin{minipage}[b]{.4\textwidth}
\includegraphics[trim=0in 0.1in 0.1in 0.in,clip,width=1.0\textwidth]{figures/prc_init.png}
\caption{Precision Recall Curve for KNN model for initial round of labels.}\label{fig:prc_init}
\end{minipage}

\centering
\begin{minipage}[b]{.4\textwidth}
\includegraphics[trim=0in 0.1in 0.1in 0.in,clip,width=1.0\textwidth]{figures/rocs_round5.png}
\captionsetup{labelformat=empty}
\caption{Receiver Operating Curve for iteration 4.\logan{I think we could benefit from easier-to-remember labels than AL1 and AL2. I don't remember which is which. Also why does the file say Round 5 and the text iterating curves? If we say ROC is a worse plot, why even show it?}}\label{fig:rocs_round5}
\end{minipage}\qquad
\begin{minipage}[b]{.4\textwidth}
\includegraphics[trim=0in 0.1in 0.1in 0.in,clip,width=1.0\textwidth]{figures/prcs_round5.png}
\captionsetup{labelformat=empty}
\caption{Precision Recall Curves for iteration 4.}\label{fig:prcs_round5}
\end{minipage}
\setcounter{figure}{8}  
\caption{Receiver Operation and Precision Recall Curves for iterations 4 and 5. AL - 1 refers to active learning using NLP-filtered nouns. AL - 2 refers to active learning using candidates deemed similar to seed entities.  \logan{<soapbox>Is it standard practice to use such uninformativ figure captions in CS? I was taught to make the captions almost standalone from the text</soapbox>}}\label{fig:rocs_prcs_round5}
\end{figure*}

\subsubsection{Discrimination}
After the initial round of labeling, we experiment with the three strategies mentioned in Section~\ref{sect:apner_architecture}: random sampling (Random), active learning using pool of NLP-filtered nouns from training corpus (UBS), and active learning using pool of \nistnum{10000} distance candidates deemed closest to our seed entities(Distance UBS) \textemdash minus 200 used in initial round).
Given the relatively small batch size, we see no improvement in the precision recall curve for two rounds of active learning or first three rounds of labeling. 
Precision remains under 5\% across all used classifiers as shown in Table~\ref{tab:pr_table}. 
\logan{I think the table would be better suited as a plot}
\loganfussingaboutrecallandprecision

\begin{table}[ht!]
\centering
\caption{Precision and recall when tested against ground truth documents for classifiers at each round of the active learning process.\label{tab:pr_table}
}
\vspace{2ex}
%[35.55, 37.69, 46.9, 47.97, 46.47, 48.39, 46.68, 36.4]
\setlength\tabcolsep{3pt}
%\begin{tabular}{*4c}
\begin{tabular}{|c|c|c|c|c|}
\hline
& & \multicolumn{3}{c|}{\textbf{Strategies}} \\
 Round \# & Metric & \textbf{Random} & \textbf{UBS}  & \textbf{Distance UBS}  \\
%\hline
%Round 0\ \ & \multicolumn{3}{c|}{} \\
\hline
0 & Precision &        \multicolumn{3}{c|}{6.53\%} \\
0 & Recall\ \ \ \ \ &               \multicolumn{3}{c|}{19.1\%} \\
%\hline
%Round 1\ \  & \multicolumn{3}{c|}{} \\
\hline
1 & Precision     & 3.75\%       &      3.18\%      &  5.28\% \\
1 & Recall\ \ \ \ \ & 0.29\%      &   93.62\%     &  56.81\% \\
%\hline
%Round 2\ \  & \multicolumn{3}{c|}{} \\
\hline
2& Precision      & 1.49\%           &      3.79\%      &  5.35\% \\
2& Recall\ \ \ \ \ & 1.45\%           &    46.38\%      &  10.14\% \\
%\hline
%Round 3\ \  & \multicolumn{3}{c|}{} \\
\hline
3 & Precision      & 6.02\%              &    21.23\%      &  3.93\% \\
3 & Recall\ \ \ \ \ & 44.64\%            &    40.00\%      &  84.35\%  \\
%\hline
%Round 4\ \  & \multicolumn{3}{c|}{} \\
\hline
4& Precision     & 7.06\%          &    18.21\%      &  7.20\% \\
4& Recall\ \ \ \ \ & 40.70\%         &     45.64\%     &  51.88\% \\
\hline
\end{tabular}
\end{table}


However in the third iteration of active learning, we notice an increase in precision in UBS (see Table~\ref{tab:pr_table}). 
This learning is sustained in the following iteration as illustrated on Figures~\ref{fig:rocs_prcs_round5}. 
The AUC for UBS is 0.74 and that of Distance UBS is 0.70. 
The PRCs for both are improved (lifting away from the lower left corner of the graph) over the first round with active learning with UBS showing better tradeoff than with Distance UBS and Random strategies.
%After round 3 and 4, we observe and confirm some, albeit moderate, level context-based predictive ability.
%We also determine that the classifier, which uses UBS is performing better than with the two other strategies.
When tested against our gold standard of 467 one-word polymer names the KNN classifier achieves 18.21\% precision and  45.64\% recall. 
While the candidate generation helps insure that the classes are balanced in the initial batch of labels, using a pool of distance candidates (Distance UBS) does not yield better results than using active learning with a pool of all NLP-filtered candidates (UBS).
Intuitively, basic UBS is able to find \textit{useful} instances (target and non-targets) to be labeled from the entire word embedding space, while examples from Distance UBS are clustered around the seed entities which may be colocated in that space.
We selected seed entities based on their frequency in our corpus, and the yield of target entities during the candidate generation.
Instead, this observation suggests that we could also study how the choice of seed entities impact of the performance of the classifier during the active learning process. 
We revisit this concept of \textit{diversity} of labels in more details in the Section~\ref{sec:discussion}. 
It is worth noting that with limited training data and based solely on context, the classifiers retrieves 45.64\% (more than one third) of the gold standard polymers with a precision of 18.21\%; this after about five hours of expert labeling. 
For comparison, an attempt to extract polymer names using the rule: \textit{if the name contains ``poly'' extract it as a polymer}, would score a precision of 34\% and a recall of 41\% on the same dataset.
\logan{I think we should show this point on the plot for context. Also, state what you conclude from the comparison. Don't leave it up to the reader to state the conclusion.}

\subsection{Using Active Learning Labels with Character-Level Enhanced Embeddings}
Next, we train a word embedding model enhanced with sub-word information using polyNER's labels. 
\logan{Why are you doing this? You need some motivation for this section, otherwise it kinds of jumps out of nowhere and begs the question: Why did you not use this through the entire study? Maybe you can say this section is devoted to "further tuning the model now that you have a large-enough pool of data to make selections statistically-reasonable"?}
We compare the performance to a state-of-the-art chemistry-aware NLP toolkit. %and we end with a discussion of the results.
FastText uses word representations enriched with character-level information.
This word embedding method considers sub-word information as well as
context, allowing it to consider word morphology differences, such as prefixes
and suffixes. Sub-word information is especially useful for words for which
context information is lacking, as words can still be compared to morphologically similar
existing words. We set the length of the sub-word used for comparison\textemdash
FastText's n_gram parameter\textemdash to five characters, based on our intuition that
many polymers begin with the prefixes ``poly'' or ``poly(.'' 
Therefore, we generate a FastText word embedding model, and generate character-enhanced vectors for our UBS-labeled candidates.


Next, we train a KNN classifier using vectors for these candidates labeled through UBS or active learning using a pool of NLP-filtered candidates (identified as best-performing in previous experiments).
\logan{Confusing. Is UBS not active learning?}
We test the classifier against NLP-filtered nouns from our 100-document test set.
The KNN classifier performance improves when using these word vectors as shown in Figures~\ref{fig:UBS_rocs_fasttext} and~\ref{fig:UBS_prcs_fasttext}.
\logan{Why sow both figures if you say PRC is better.}
In this case, the classifier achieves 29.7\% precision and 81.9\% recall. 
These numbers are comparable to those achieved by ChemDataExtractor (CDE), a state-of-the-art chemical NLP tool.
As CDE aims to extract all
chemical compounds, not just polymers, it serves only as a demonstration of an
alternative approach in the absence of a polymer NER system (Note that CDE also extracts properties). 
Its recall is high
at 74.5\% but its precision is, as expected, low at 8.7\%. 
\loganfussingaboutrecallandprecision
We have previously modified CDE with
manually defined polymer identification rules~\cite{tchoua2017towards},
and our polymer-enhanced version of the software (CDE+) achieved 42.2\% precision and 68.3\% recall on the same test set. 
We achieve higher recall than CDE and CDE+ using labels from UBS and FastText vectors and intermediate precision with only $\sim$ five hours of expert labeling.

\subsection{Discussion}
\label{sec:discussion}
\logan{The discussion starts out with your last part of the Results section, skipping over anything else you learned. }
We attribute the increased performance to the character embedding enhancement, which not only recognizes ``poly'' (and yields more names based on this n-gram comparison), but also filters out more anomalous candidates (preceding or following polymer names) generated during tokenization and missed by the filtering steps such as ``\textit{$A_mB_n$}'', or ``\textit{Mw/Mn=1.36}''.  
In other words, the classifiers of character and (context-based) word embedding vectors performs better than classifiers of only context-based word embedding.
Given this result, one may wonder whether the active learning process itself could benefit from using this enhanced vector embedding. 
We repeated the active learning experiment using the entire corpus of NLP-filtered candidates and classifying FastText (enhanced) vectors instead of Gensim vectors at each iteration. 
The classifier's ROC curve did not achieve CDE performance after 5 iterations.\logan{Wasn't it also worse than the randomly-selected words?}
Theese results suggest that the character-level information enhances the classifier's performance only once a certain threshold of context information has been captured by the embedding.
In FastText, the portion of the word embedding vector generated using context varies depending on how much context is available in the entire corpus. 
For words deemed to have \textit{enough} context, vectors do not include any character-level information. 
At the other extreme, for previously unseen words, the embedding is generated solely based on character n-gram information and comparison to other words in the corpus.
During the active learning process, candidates to be labeled by experts are selected using maximum-entropy based uncertainty sampling (for which the prediction probability is the same for both classes, target and non target);
these are also more likely to be candidates which lack context and for which vectors have been generated using character-level information. 
As a result, the expert is presented with several nearly identical candidates (e.g. \textit{PS13k/PMMA12k}, \textit{PS214k/PMMA12k}, and \textit{PS31.6k/PMMA12k}) hindering the learning process as they are located in close vicinity in terms of the full (character and context) word embedding space.
In other words, in this full space, while their uncertainty measure is comparable, these examples are not \textit{diverse}, where diversity is a measure of the distance of the examples to each other or previously labeled instances~\cite{brinker2003incorporating}.
One solution to explore in the future, would be to impose a diversity constraint on the candidates using batch active learning for example.\logan{There are batch active learning papers, and surely a review. Cite one so that people do not think you're planning to reinvent such methods.}
\logan{This seem more like a "Future Work/Can we do even better!?" section than an actual discussion. Should we rename the section accordingly}


%\begin{figure}[H]
\begin{figure}
\centering
\begin{minipage}[b]{.4\textwidth}
\includegraphics[trim=0in 0.1in 0.1in 0.in,clip,width=1.0\textwidth]{figures/fasttext_roc_al_corpus_round5_100}
\caption{Receiver Operating Curve for KNN model trained using active learning labels and word representations enriched with character-level information.}\label{fig:UBS_rocs_fasttext}
\end{minipage}\qquad
\begin{minipage}[b]{.4\textwidth}
\includegraphics[trim=0in 0.1in 0.1in 0.in,clip,width=1.0\textwidth]{figures/fasttext_prc_al_corpus_round5_100}
\caption{Precision Recall Curve for KNN model trained using active learning labels and word representations enriched with character-level information.}\label{fig:UBS_prcs_fasttext}
\end{minipage}
\end{figure}












%------------------------------------------------------------------------------

%------------------------------------------------------------------------------
\section{Conclusion}
\label{sect:apner_conclusion}
The lack of access to large amount of expert-annotated training data impedes the adoption of recent machine learning techniques in certain scientific applications.
PolyNER is a generalizable system that can efficiently use expert input via approximate candidate generation and active learning for scientific named entity recognition.
We show that using natutal language processing techniques, we can bootstrap a word vector classifier of scientific entities.
Using polyNER's labels and a classifier of character-enhanced word embedding vectors, we achieve 
%achieves 31.7\% precision and 82.3\% recall,  a 
performance comparable to best-of-breed
%\loganfussingaboutrecallandprecision \roselyne{came with PR curve showing good trade off margins}
hybrid NER model (CDE+) that combines a dictionary, expert created
rules, and machine learning algorithms.
CDE+ was trained on the CHEMDNER corpus:
a collection of \nistnum{10000} PubMed abstracts that contain a total of 84,355 chemical entity mentions labeled manually by expert chemistry literature curators, following annotation guidelines specifically defined for this task~\cite{krallinger2015chemdner}. 
%\logan{Very long sentence, break it up.}
By contrast, polyNER was trained on data annotated using $\sim$ five hours of expert time and minimal untrained crowd input.
%\logan{This is your most important conclusion, perhaps bring it earlier in the conclusion}
While we note that CDE+ is intended to extract polymers and their properties, our work highlights the potential of using minimal amount of data and focused expert input in order to enable machine learning techniques for previously unmined scientific entities. 
We are currently exploring using polyNER-labeled data to annotate text for other NER approaches,
such as bidirectional long short-term memory models.
We plan to explore the generalizability of our approach further by applying it to different field, that is extracting another previously unmined scientific entities.
%With a view to exploring generalizability, we are also working to apply polyNER
%to quite different problems, such as extracting dataset names from social science
%literature. 
%\logan{Maybe be a little more broad here. "We hope that our findings will ...", etc.}\roselyne{this is also too close to previous conclusion, hadn't worked on it enough yet}
%------------------------------------------------------------------------------


% use section* for acknowledgment
\section*{Acknowledgments}
This work was supported in part by NIST contract 60NANB15D077, the Center for Hierarchical Materials Design, and DOE contract DE-AC02-06CH11357, and by computer resources provided by Jetstream~\cite{towns2014xsede,stewart2015jetstream}.
Official contribution of the National Institute of Standards and Technology; 
not subject to copyright in the United States.

\balance
\bibliographystyle{IEEEtran}
\bibliography{references}

% that's all folks
\end{document}
