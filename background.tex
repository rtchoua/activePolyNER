\section{Motivation}
\label{sect:background}
%\roselyne{Adjust/shorten since I added text.}
%\kyle{Sounds similar to previous paper. First paragraph should be re-written.}
%\roselyne{Yes, it's exactly the same! Made a note to myself but hadn't come back to it yet.}
%\ian{This is the first reference to ``chemical NER,'' which you have not previously said you are addressing: 
%above, you refer to materials science.}\roselyne{mistake, I think I can directly say polymer science here, or have a different intro}
Our work is generally motivated by the need to extract previously unmined scientific entities;
our initial goal is to enable machine-learned extraction of polymer names. 
The challenges of polymer science NER are similar to those in biomedicine~\cite{krallinger2015chemdner,kim2004introduction}. 
Entities can be described with multiple referrents (synonymy).
Conversely, the same word may refer to different concepts depending on context (polysemy).
For example,
\textit{polystyrene} is often referred to as \textit{PS}, but \textit{polystyrenes} can also be referred to as \textit{GPPS}, \textit{HIPS}, and \textit{EPS}; combined with other monomers yielding \textit{SBR}, \textit{SBS}, and \textit{ABS}; or used to describe polystyrene derivatives such as \textit{PAMS}, \textit{PMS}, and \textit{PSS}.
%\textit{Polystyrene} is often referred to as \textit{PS}, but \textit{polystyrenes} also describes a family of polymers including \textit{GPPS}, \textit{HIPS}, \textit{EPS}, \textit{SBR}, \textit{SBS}, and \textit{ABS}. 
%\roselyne{Ask Debbie to check previous example. Done!}
While standards for naming polymers exist (e.g., International Union of Pure and Applied Chemistry (IUPAC~\cite{hiorns2013brief}) naming conventions), they are not always followed in practice~\cite{tamames2006success}. 
Instead, polymer names may be reported as source-based names (based on the monomer name), structure-based names (based on the repeat unit), common names (requiring domain specific knowledge), trade names (based on the manufacturer), and names based on chemical groups within the polymer (requiring context to fully specify the chemistry), generating variability in naming conventions.
Typographical variants (e.g., alternative uses of hyphens, brackets, spacing) and alternative component orders cause more variations between polymer names in the literature.
The origin of these different naming conventions is linked to the desire for clarity within a journal article, coupled with the often complicated monomeric structures~\cite{audus2017polymer}.
%For example, sequence-defined polymers, where multiple monomers are chemically bound in a well-defined sequence as in proteins, often defy normal naming practices as it is not possible to concisely list every monomer and the respective position of the monomer~\cite{lutz2014polymer}. 
%Another class of polymers that often suffer from complicated names are conjugated polymers, which exhibit useful optical and electrical properties. 
%Conjugated polymers are complex due to the co-polymerization of multiple monomers (donor/acceptor units), the type and position of side chains along the polymer backbone and the coupling between monomer units to control regioregularity~\cite{himmelberger2015engineering}.

In addition to challenges related to the makeup of the scientific entities, the scarcity of entities in scientific literature and the lack of training data also impedes the use of recent machine learning based NER techniques.
%\logan{I don't know the word "paucity." Am I just unread, or should we use a lower-brow word?}
%\roselyne{I need to admit you made me google the term lower-brown word :-) will use scarcity, or sparseness or rarity?}
Considerable time and manual effort are involved in creating and maintaining the balanced
CoNLL dataset for standard NER~\cite{tjong2003introduction}.
Example sentences from such corpora include one or more entities per sentence. 
In our attempt to recognize polymer names in full text documents, we face a very imbalanced dataset where most sentences do not contain a target entity, as there are only a handful of target entities per document.
%(we found less 2\% of words in sample publications are polymers in our test dataset)
While there has been much interest in machine learned recognition of biochemical entities~\cite{jessop2011oscar4,rocktaschel2012chemspot,leaman2015tmchem,swain2016chemdataextractor}, 
the successes that have been achieved have required much human effort to generate quality training data~\cite{krallinger2015chemdner}.
%\ian{``there has concurrently been significant effort involved in selecting...'': 
%can we say simply, ``the successes that have been achieved have required much human effort to
%generate quality training data~\cite{krallinger2015chemdner}.''? 
%The current text seems to me to be oblique and also to imply that it may be possible to generate training
%data \emph{without} manual effort.}
%there has concurrently been significant effort involved in selecting and (often manually) generating quality data for
%trainable statistical NER systems~\cite{krallinger2015chemdner}. 
Previous work has also found that even state-of-the-art NER systems do
not typically perform well when applied to different domains~\cite{krallinger2013overview}. 
Therefore the problem of training machine learning models to recognize new scientific entities in a new field, such as polymer science, remains challenging.



