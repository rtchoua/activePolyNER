\section{Introduction}
\label{sect:apner_introduction}
A wealth of valuable research data is published in unstructured form in millions of scientific articles each year. %\cite latest version of that report?"
Reading and extracting pertinent information from those articles has
become an unmanageable task for scientists and makes
it hard to build on existing results. 
%In many fields of science, the scientific literature contains large quantities of important data in unstructured form that,
%if extracted, could benefit research.
A major obstacle to scientific fact extraction is the difficulty of identifying scientific entities in text.
Despite much progress in natural language processing (NLP), 
scientific named entity recognition (NER) remains a research challenge.
The main reason for this gap between NLP advances and scientific extraction needs is the lack of carefully annotated datasets for specific targets.
In standard NER, 
progress is made possible by, for example, the 
Conference on Computational Natural Language Learning (CoNLL) dataset,
which supports much work that advances the state of the art.
But NER systems trained on CoNNL data do not perform well for scientific text, due to 
the distinctive vocabularies used in different scientific disciplines and subdisciplines.

Science-specific training datasets have been established in
%Efforts have been launched 
%In contrast, there is a conference organized around the CoNLL dataset for standard NER and other NLP tasks to push the state-of-the-art each year.
%Driven by bioinformatics\textemdash which develops methods and software tools for understanding biological data\textemdash similar efforts exist in 
biology~\cite{song2004posbiotm} and, more recently, chemistry~\cite{krallinger2015chemdner}.
However, the expert effort required to 
design 
%complete 
extraction schema, 
define clear annotation rules, and generate training data % for most advanced machine-learning based NLP techniques 
is substantial, and cannot feasibly be performed for 
%has to be repeated for 
every field of science.
%\logan{Are you sure you want to use materials science as the opening line? Maybe add a general sentence up top "data-intensive science is awesome, but requires accessible data. The development of new materials is a great example for such opportunities and needs in domain science. [I'm making these comments under the context that this will be a thesis chapter]}\roselyne{point taken!}
As a consequence, annotated datasets do no exist in most fields,
%In many fields, such as material science, no such annotated datasets exist.
preventing the application of state-of-the-art NER and fact extraction methods.

%This problem is apparent in materials science where, 
%despite many publications containing much data,
%a lack of annotated datasets impedes progress in materials informatics, 
%a field that aims to replace current trial-and-error materials design processes
%by combining large datasets and computational models to achieve targeted materials design, 
%thus reducing time-to-market and development costs of new materials.

This problem is apparent in materials science, where materials informatics 
seeks to combine large datasets and computational models to 
replace current trial-and-error materials design processes with targeted materials design, 
thus reducing time-to-market and development costs of new materials~\cite{MGI2014}.
A lack of annotated training data has hindered progress,
preventing large-scale application of methods pioneered in, for example,
biomedicine.
Those methods, which often use hybrid rule-based, machine learning, and statistical techniques to extract entity names and relations from the literature~\cite{leaman2008banner,zeng2015survey}, require much training data.
%the process of designing new materials is still one of trial and error.
%The field of materials informatics aims to address this problem by combining large datasets and computational models to achieve targeted materials design and reduce time-to-market and development costs of new materials.
%A large amount of such data already exists in unstructured text in the scientific literature, but the lack of training data impedes the direct application of state-of-the-art extraction methods.
% hence the need for materials information extraction.
%\logan{Good intro paragraph. "Opportunity -> problem you are solving" in an inch of text}\roselyne{Thanks!|}
%There is a considerable amount of prior work in scientific facts extraction, notably in biomedicine. 
%State-of-the-art methods often use hybrid rule-based, machine learning, and statistical techniques to extract entity names and relations from the literature~\cite{leaman2008banner,zeng2015survey}. 
While similar efforts have begun in materials science~\cite{hawizy2011chemicaltagger,rocktaschel2012chemspot,leaman2015tmchem,swain2016chemdataextractor,young2018data}, the lack of available training data for new materials impedes rapid progress. 
Instead, each new research project targeting a new type of materials must first undertake considerable
effort to create a large, carefully annotated training
data tailored for this new target, a task that often requires considerable in-depth domain knowledge.

The subfield of polymer science puts these problems in particularly stark relief.
Polymers have their own unique nomenclature, as we explain below, and thus annotated datasets created for general
chemistry are of little value.
Scientists and engineers lack access to a freely available large database of polymers and their properties.
%\kyle{should cite our previous papers about polyNER. ANd change phrasing here to make it sound like polyner is now well established
%and what we are doing here is enhancing it with active learning. }
To address these challenges, we have previously designed polyNER~\cite{tchoua2018towards,tchoua2019polyner}, a system for generating training data for scientific NER using semi-supervised and supervised learning.
PolyNER uses NLP to produce sets of candidate entities, which
experts 
%iteratively 
approve or reject via a 
%easy-to-use 
Web interface;
%in small batches selected using an active learning
%\ian{You say that polyNER uses active learning, then you say that you enhance it with AL in this paper. 
%Inconsistent?}\roselyne{yes thank you, made some changes (too fast) after Kyle pointed out that we had already introduced polymer in the now accepted paper}.
% approach;
the resulting labels are used to train context-based word vector classifiers. %after each round.
PolyNER's labels can also be used to train other machine learning models to leverage other features in addition to context, such as word morphology
to recognize target entities.
%\ian{The reference to ``beyond their context'' doesn't make sense to me; I wonder if it will to other readers.}
The goal is
to substitute the labor-intensive processes of assembling a large
manually annotated corpus (and reduce costs) by using small numbers of carefully selected candidates to be labeled via focused expert input. 
We aim in our work to slash the expert time and effort required to achieve state-of-the-art NER performance
in a new field.

In this paper, we seek to improve polyNER's performance by improving the labeling process and the classification of entity word vectors. 
Specifically, we address two challenges: 
(1) lack of (expensive) training data in some fields, including our own polymer science application which lacks free access to large polymer databases; and 
(2) the need for domain expert curators, which impedes the use of crowdsourcing platforms such as Amazon Mechanical Turk~\cite{buhrmester2011amazon} or Figure8~(\url{https://www.figure-eight.com/}).
%\ian{I note that in the architecture section, you refer to this phase as sampling, so I changed 
%the text in the results section to be consistent. Not sure if you want to change here too, or change other secitons back,
%but should be consistent.}
%labeling, and classification.\ian{Classification is not shown as a step elsewhere. Also, I note that Figure~1 has a
%``candidate discrimination'' step, which is a term not used much elsewhere.}\roselyne{Noted! I changed figure 1 and will check that we don't use discrimination elsewhere. Also remove any mention of the candidate generation since now we only do it at the initial labeling.}
In the initial labeling phase, 
we experiment with different \textit{representative} (commonly used) entities to increase the fraction of target entities in the dataset to be labeled and bootstrap the context-based word vectors classifiers. %\ian{I think that the representative/seed entity thing is mainly relevant for bootstrapping. Maybe we should say ``used in the bootstrapping phase.'''}
 % to generate candidates;
% and tune the word embedding model parameters to increase the yield of actual scientific entities among candidates; 
In the subsequent labeling and classification steps,
%\ian{is this rewording correct? I think that as you say you improve each step,
%you should be explicit about what you did to improve in each step.}\roselyne{yes correct}
we use active learning with maximum entropy uncertainty sampling and two different pools of unlabeled data to train classifiers, and compare their learning rate after five rounds. 
%\logan{"seed entities" is not a standard term, right? If it is not standard, maybe use something less metaphorical "ways to generate initial candidates for labeling"}\roselyne{I don't think so standard but I picked it up from similar works in NER, snowball, do you mean not to use it in the intro or the whole paper. Will search and see how many times I use it. I had started using representative entities, which I also have to define.} 
Using labels generated via active learning, we train word vector classifiers and achieve NER performance comparable to 
a modified version of a state-of-the art rule-based chemical entity extraction
system, ChemDataExtractor (CDE)~\cite{swain2016chemdataextractor}.
We have previously enhanced CDE
with dictionary- and rule-based methods for identifying polymers~\cite{tchoua2017towards}.
%\ian{It is not clear, I think, whether you are comparing against the regular CDE or the enhanced CDE. Should you compare against both?}\roselyne{CDE is extracting all chemicals, we actually achieve higher recall which could be an important result, but doesn't it complicate explanations?}
Our system, however, took five hours of expert time to achieve this result.
%\ian{We mention five hours repeatedly, sometimes saying five hours, sometimes less than five hours, sometimes $sim$ five hours. I think we should mention it less often, and be consistent in how we phrase it.}

The rest of this paper is as follows. 
In Section~\ref{sect:background}, we motivate the need for identifying polymer names in
text. 
We review semi-supervised methods for NLP systems in
Section~\ref{sect:apner_related}. 
We describe the design and implementation in Section~\ref{sect:apner_architecture} and evaluate polyNER
in Section~\ref{sect:apner_results}. We summarize and discuss future work in Section~\ref{sect:apner_conclusion}.
%\logan{<soapbox>I only see CS have this "table of contents" paragraph. Is it explicitly required, or just an idiosyncrasy of the CS community? Also, does everyone skip over them while reading anyway?</soapbox>}\roselyne{Very common in computer science paper I think, I'll let Ian and Kyle confirm.}




%\roselyne{Moved this to intro and used some of the text as it is repeated in background bu may be clearer than current text in intro}
%\ian{I edited the following a bit to avoid what seemed to me to be repetition with preceding sections.
%One concern: I wonder if the end of this paragraph repeats (perhaps more clearly!) what we say in the introduction
%about the goals of the paper?}
%We aim in our work to slash the expert time and effort required to achieve state-of-the-art NER performance
%in a new field.
%We have previously shown that with just two hours of expert labeling effort, 
%polyNER can achieve  
%precision or recall on par with a well-performing hybrid NER model
%that combines a dictionary, expert-created rules, and machine learning algorithms~\cite{tchoua2019polyner}. 
%In this paper, we describe how we further improve polyNER's performance by using active learning.

%\roselyne{Again moved this to intro to see if I can use it here instead of in related work}
%PolyNER extends previous work to create a low-cost scientific fact extraction system. 
%We ask: how can we quickly generate annotated data for scientific named entity recognition?
%Specifically, we address two challenges: 
%(1) lack of (expensive) training data in some fields, including our own polymer science application which lacks free access to large polymer databases; and 
%(2) the need for domain expert curators, which impedes the use of crowdsourcing platforms such as Amazon Mechanical Turk~\cite{buhrmester2011amazon} or Figure8~(\url{https://www.figure-eight.com/}).